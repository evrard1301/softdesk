\section{Sécurité}

\begin{frame}{Sécurité}  
\end{frame}

\begin{frame}{Enjeux de la sécurité}

  \begin{block}{Tout est une question de \underline{risques}}
    Risque = un impact x probabilité d'occurrence
  \end{block}

  \begin{block}{Rédaction d'un modèle de menaces (\textit{threats modelling})}
    Quatre questions à se poser\footnote{\url{https://owasp.org/www-community/Threat_Modeling}}:
    \begin{itemize}
    \item Sur quoi travaillons-nous ?
    \item Qu'est-ce qu'il pourrait arriver de mauvais ?
    \item Quoi faire si cela arrive ?
    \item Avons-nous fait du bon travail ?
    \end{itemize}
  \end{block}
\end{frame}

\begin{frame}{OWASP 2017}
  \begin{block}{Le top 10 des menaces de sécurité}
    \begin{enumerate}
    \item TODO enumeration des menaces
    \end{enumerate}
  \end{block}
\end{frame}

\begin{frame}{Sécurité mise en place par Django}
  TODO: qu'est-ce que django fait automatiquement ?
  \newline
  TODO: quoi configurer ?
  \newline
  TODO: comment configurer ?
\end{frame}

\begin{frame}{Renforcer la sécurité}
  \begin{block}{Sécurité supplémentaire}
    \begin{itemize}
    \item Authentification par nom d'utilisateur et mot de passe:
      \begin{itemize}
      \item Vérification de la sécurité du mot de passe (TODO: voir django password validator)
      \end{itemize}
    \item Mise en place d'un système de rôles:
      \begin{itemize}
      \item Administrateur
      \item Auteur d'un projet, d'un problème ou d'un commentaire
      \item Collaborateur
      \item Utilisateur
      \end{itemize}
    \end{itemize}
  \end{block}
\end{frame}

\begin{frame}{Tester la sécurité de l'API}
  \begin{block}{Différents types de tests}
    \begin{itemize}
    \item \textit{Scan} de vulnérabilités
    \item Tests de pénétrations
    \item Audits de sécurité
    \item ...
    \end{itemize}

    Comment tester la sécurité de l'API ?
  \end{block}

  \begin{block}{$\rightarrow$ \textit{Via} les tests de Django}
    TODO: différents niveau de tests, ici on parle de tests non fonctionnels d'acceptations
    et non pas de tests fonctionnels unitaires ($\rightarrow$ en complément, non pas à la place).
  \end{block}
\end{frame}

