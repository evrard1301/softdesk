\section{Documentation}

\begin{frame}{Documentation}  
\end{frame}

\begin{frame}{Pourquoi documenter ?}
  \begin{block}{Plusieurs raisons de maintenir la documentation d'un projet}
    \begin{itemize}
    \item TODO
    \end{itemize}
  \end{block}
\end{frame}

\begin{frame}{Quoi documenter ?}
  \begin{block}{Tout au long du cycle de vie du logiciel}
    \begin{itemize}
    \item Les exigences (cahier des charges)
    \item Le développement (dans le code source et la documentation externe)
    \item Les tests (plan de test, stratégie de test, ...)
    \item Les analyses et audits
    \item Le manuel pour utilisateurs ou développeurs
    \end{itemize}
  \end{block}
\end{frame}

\begin{frame}{Documentation et développement}
  \begin{block}{\textit{Documentation Driven Development}}
    \begin{itemize}
    \item Mise en place d'une documentation \textbf{avant} l'implémentation.
    \end{itemize}
  \end{block}

  
  \begin{block}{Documentation de l'API}
    \begin{itemize}
    \item Documentation \textit{via} un wiki directement depuis \textit{github}.
    \item Utilisation de \textsf{sphinx}
    \end{itemize}
  \end{block}  
\end{frame}

\begin{frame}{Documenter le code}
  \begin{block}{Sphinx}
    \begin{itemize}
    \item Utilisée par \textsf{read the docs}
    \item Permet de créer une documentation à partir de fichiers
      \textsf{reStructuredText}.
    \end{itemize}
  \end{block}  
\end{frame}

\begin{frame}[fragile]{Documentation: Sphinx}
  \tiny
  \begin{center}
    \begin{figure}
      \begin{lstlisting}
        This is a Title
        ===============
        That has a paragraph about a main subject and is set when the '='
        is at least the same length of the title itself.

        Subject Subtitle
        ----------------
        Subtitles are set with '-' and are required to have the same length
        of the subtitle itself, just like titles.

        Lists can be unnumbered like:

        * Item Foo
        * Item Bar

        Or automatically numbered:

        #. Item 1
        #. Item 2

        Inline Markup
        -------------
        Words can have *emphasis in italics* or be **bold** and you can define
        code samples with back quotes, like when you talk about a command: ``sudo``
        gives you super user powers!
      \end{lstlisting}      
      \caption{Exemple de document rédigé \textit{via} la syntaxe \textsf{reStructuredText}}      
    \end{figure}
  \end{center}
\end{frame}
