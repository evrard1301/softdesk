\section{Développement}

\begin{frame}{Développement}  
\end{frame}

\begin{frame}{Vue d'ensemble}
  \begin{block}{Applications}
    \begin{itemize}
    \item \textsf{authentication}: pour gérer les utilisateurs et
      l'authentification.
    \item \textsf{project}: pour gérer les projets et ce qui tournent
      autour.
    \end{itemize}
  \end{block}
\end{frame}

\begin{frame}[fragile]{L'authentification}
  \begin{itemize}
  \item Authentification par jeton JWT.
  \item Utilisation de la bibliothèque
    \textsf{djangorestframework-simplejwt}.
  \item Fournit une vue toute faite \textsf{TokenObtainPairView}.
  \end{itemize}

  \begin{center}
    \tiny
    \begin{lstlisting}[language=python]
      ...
      from rest_framework_simplejwt.views import TokenObtainPairView
      
      urlpatterns = [
          ...
          path('login/', TokenObtainPairView.as_view(), name='login')    
      ]
    \end{lstlisting}
  \end{center}
\end{frame}

\begin{frame}{Les rôles}
  \begin{block}{Plusieurs rôles possibles}
    \begin{itemize}
    \item Collaborateur d'un projet:
      \begin{itemize}
      \item Contributeur (\textit{contributor})
      \item Responsable (\textit{supervisor})
      \item Auteur (\textit{author})
      \end{itemize}
    \item Un utilisateur peut avoir plusieurs rôles !
    \end{itemize}
  \end{block}
\end{frame}

\begin{frame}[fragile]{Modèle d'un collaborateur}
  \begin{center}
    \tiny
    \begin{lstlisting}[language=python]
    class Collaborator(models.Model):
      SUPERVISOR_ROLE = 'SUPERVISOR'
      CONTRIBUTOR_ROLE = 'CONTRIBUTOR'
      AUTHOR_ROLE = 'AUTHOR'
      
      ROLES = [
        (SUPERVISOR_ROLE, 'supervisor'),
        (CONTRIBUTOR_ROLE, 'contributor'),
        (AUTHOR_ROLE, 'author')
      ]
      
      user = models.ForeignKey(User, on_delete=models.CASCADE)
      project = models.ForeignKey(Project, on_delete=models.CASCADE)
      role = models.CharField(max_length=256, choices=ROLES)
    \end{lstlisting}
  \end{center}
\end{frame}

\begin{frame}[fragile]{Les projets}
  \begin{block}{Un projet c'est:}
    \begin{itemize}
    \item Un titre.
    \item Une description.
    \item Un type: ANDROID, IOS, FRONTEND ou BACKEND.
    \end{itemize}
  \end{block}

  \begin{center}
    \tiny
    \begin{lstlisting}[language=python]
  class Project(models.Model):
    ANDROID_TYPE = 'ANDROID'
    IOS_TYPE = 'IOS'
    BACKEND_TYPE = 'BACKEND'
    FRONTEND_TYPE = 'FRONTEND'
    
    TYPES = [
        (ANDROID_TYPE, 'Android'),
        (IOS_TYPE, 'IOS'),
        (BACKEND_TYPE, 'Back-end'),
        (FRONTEND_TYPE, 'Front-End')
    ]
    
    title = models.CharField(max_length=256)
    description = models.CharField(max_length=4096)
    type = models.CharField(max_length=64, choices=TYPES)
    \end{lstlisting}
  \end{center}
\end{frame}

\begin{frame}[fragile]{Modèle d'un problème}
  \begin{center}
    \tiny
    \begin{lstlisting}[language=python]
      class Issue(models.Model):
        # ...
      
        title = models.CharField(max_length=256)
        description = models.CharField(max_length=4096)
        tag = models.CharField(max_length=128, choices=TAGS)
        priority = models.IntegerField()
        status = models.CharField(max_length=128, choices=STATUS)
        project = models.ForeignKey(Project, on_delete=models.CASCADE)
        author = models.ForeignKey(
            User,
            on_delete=models.CASCADE,
            related_name='author'
        )
        assignee = models.ForeignKey(
            User,
            on_delete=models.CASCADE,
            related_name='assignee'
        )
        created = models.DateTimeField(auto_now=True)
    \end{lstlisting}
  \end{center}
\end{frame}

\begin{frame}[fragile]{Tags et status}
  \begin{center}
    \tiny
    \begin{lstlisting}[language=python]
      class Issue(models.Model):
        BUG_TAG = 'BUG'
        IMPROVEMENT_TAG = 'IMPROVEMENT'
        TASK_TAG = 'TASK'
        
        TAGS = [
            (BUG_TAG, 'bug'),
            (IMPROVEMENT_TAG, 'improvement'),
            (TASK_TAG, 'task')
        ]
        
        OPEN_STATUS = 'OPEN'
        CLOSED_STATUS = 'CLOSED'
        
        STATUS = [
            (OPEN_STATUS, 'open'),
            (CLOSED_STATUS, 'closed')
        ]

        # ...
    \end{lstlisting}
  \end{center}
\end{frame}

\begin{frame}[fragile]{Modèle d'un commentaire}
  \begin{center}
    \tiny
    \begin{lstlisting}[language=python]
      class Comment(models.Model):
        description = models.CharField(max_length=4096)
        author = models.ForeignKey(User, on_delete=models.CASCADE)
        issue = models.ForeignKey(Issue, on_delete=models.CASCADE)
        created = models.DateTimeField(auto_now=True)
    \end{lstlisting}
  \end{center}
\end{frame}


\begin{frame}[fragile]{Sérialisation des modèles}
  \begin{block}{Motivation}
    \begin{itemize}
    \item Permet de passer d'une classe python au format JSON.
    \item Utilisation de la classe \textsf{ModelSerializer} de DRF.
    \end{itemize}        
  \end{block}

  \begin{center}
    \tiny
    \begin{lstlisting}[language=Python]
      class ProjectSerializer(ModelSerializer):
        class Meta:
            model = models.Project
            fields = [
                'id',
                'title',
                'description',
                'type'
            ]
    \end{lstlisting}
  \end{center}
\end{frame}

\begin{frame}[fragile]{Les vues}
  \begin{block}{Les vues DRF}
    \begin{itemize}
    \item Représentent une série de \textit{endpoints} de l'API.
    \item Il y a une vue par type de ressource (projets, problèmes, commentaires etc).
    \end{itemize}
  \end{block}
\end{frame}

\begin{frame}[fragile]{Exemple: la vue des commentaires}
  \begin{block}{Définition}
    \begin{center}    
      \tiny
      \begin{lstlisting}[language=python]
        class CommentView(mixins.CreateModelMixin,
                          mixins.UpdateModelMixin,
	                  mixins.DestroyModelMixin,
                          mixins.ListModelMixin,
                          mixins.RetrieveModelMixin,
	                  viewsets.GenericViewSet):
          serializer_class = serializers.CommentSerializer
          queryset = models.Comment.objects.all()
      \end{lstlisting}
    \end{center}
  \end{block}
\end{frame}


\begin{frame}[fragile]{Exemple: la vue des commentaires}
  \begin{block}{Permissions}
    \begin{center}    
      \tiny
      \begin{lstlisting}[language=python]
        class CommentView(mixins.CreateModelMixin,
                          mixins.UpdateModelMixin,
	                  mixins.DestroyModelMixin,
                          mixins.ListModelMixin,
                          mixins.RetrieveModelMixin,
	                  viewsets.GenericViewSet):

          # ...
                          
          def get_permissions(self):
	    if self.action in ['update', 'destroy']:
                return [
                    rest_permissions.IsAuthenticated(),
                    permissions.IsProjectRelated(),
                    permissions.IsCommentAuthor()
                ]
            
            return [
                rest_permissions.IsAuthenticated(),
                permissions.IsProjectRelated()
            ]                
      \end{lstlisting}
    \end{center}
  \end{block}
\end{frame}

\begin{frame}[fragile]{Les types de permissions}
  \begin{block}{Types}
    \begin{itemize}
    \item IsProjectAuthor
    \item IsProjectRelated
    \item IsIssueAuthor
    \item IsCommentAuthor
    \end{itemize}
  \end{block}

  \begin{block}{Exemple}
    \begin{center}
      \tiny
      \begin{lstlisting}[language=python]
        class IsProjectAuthor(BasePermission):
          def has_permission(self, request, view):
            user = request.user
            project = get_project(view)
            
            role = models.Collaborator.AUTHOR_ROLE
            return models.Collaborator.objects.filter(user=user,
                                                      project=project,
	                                              role=role).count() > 0
      \end{lstlisting}
    \end{center}
  \end{block}
  
\end{frame}

\begin{frame}{URL et routage}
  \begin{block}{Django Rest Framework}
    \begin{itemize}
    \item Permet de définir des routeurs permettant de servir une
      série d'URL directement à partir d'une vue DRF.
    \item \textbf{Ne permet pas} de gérer facilement les ressources imbriquées.
    \item La documentation de
      DRF\footnote{\url{https://www.django-rest-framework.org/api-guide/routers/\#drf-nested-routers}}
      évoque une bibliothèque supplémentaire permettant justement de
      gérer ce cas de figure.
    \end{itemize}
  \end{block}
\end{frame}

\begin{frame}[fragile]{URL et routage}
  \begin{block}{Django Rest Framework Nested Routers}
    \begin{itemize}
    \item Permet de définir ressources imbriqués \textit{via} un
      routeur \textsf{NestedSImpleRouter}.
    \end{itemize}
  \end{block}

  \begin{center}
    \tiny
    \begin{lstlisting}[language=python]
      projects = routers.SimpleRouter()
      projects.register('projects', views.ProjectView, basename='projects')
      
      users = routers.NestedSimpleRouter(projects, 'projects',
                                         lookup='project')
      users.register('users',
                     views.UserView,
                     basename='users')
      
      issues = routers.NestedSimpleRouter(projects, 'projects', lookup='project')
      issues.register('issues', views.IssueView, basename='issues')
      
      comments = routers.NestedSimpleRouter(issues, 'issues', lookup='issue')
      comments.register('comments', views.CommentView, basename='comments')
    \end{lstlisting}
  \end{center}
\end{frame}

\begin{frame}[fragile]{Définitions des routes}
  \begin{block}{Les routes}
    \begin{center}
      \tiny
      \begin{lstlisting}[language=python]
        urlpatterns = [
          path('', include(projects.urls)),
          path('', include(users.urls)),
          path('', include(issues.urls)),
          path('', include(comments.urls)),
        ]
      \end{lstlisting}
    \end{center}
  \end{block}
\end{frame}
